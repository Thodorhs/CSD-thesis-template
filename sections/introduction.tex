\section{Introduction}
Today, cloud infrastructures running data-intensive applications encounter numerous challenges in transferring data to-from block storage devices. With the escalating demands of big data frameworks, the insufficiency of DRAM in scaling to such magnitudes becomes apparent. Additionally, traditional approaches to accessing local storage devices struggle to match the pace of data processing, thereby becoming an overhead. Consequently, the rapid expansion of data necessitates a swift and efficient mechanism for data movement to storage devices. 

In this \textbf{comparative analysis}, we delve into possible strategies for achieving faster and more efficient block storage accessing while ensuring an appropriate size of the block storage. The first aspect of our investigation centers on comparing local storage devices, exemplified by NVMe SSDs, against remote storage devices and remote DRAM memory. NVMe SSDs, offer rapid data access and reliability. However, the efficiency of local storage devices in comparison to remote options remains a question mark and forms the core of our study. Remote DRAM memory, despite its remote positioning, may offer lower latency compared to remote storage devices due to its faster access times. However, its capacity is typically more constrained than storage devices. It's crucial to note the role of technological advancements such as SPDK NVMe over Fabrics (NVMe-oF) in mitigating latency concerns. SPDK NVMe-oF holds promise in reducing device latency, offering a potential solution to bridge the performance gap between different storage modalities. 

By employing micro-benchmarks we try to discern the inherent trade-offs. This evaluation is crucial for comprehensively understanding the landscape of storage solutions and their applicability in real-world scenarios. To assess the performance of block device setups, we leverage TeraHeap. TeraHeap extends the capabilities of the JVM and utilizes a second heap stored on a storage device. This approach allows for a meticulous evaluation of storage solutions, providing nuanced insights into their performance characteristics. Our evaluation process, empowered by TeraHeap, furnishes high-level results that offer valuable insights into the comparative advantages and limitations of each storage solution.

The selection among these options requires careful consideration of factors such as workload requirements and infrastructure configurations. The insights acquired by this study will play a pivotal role in informing decision-making processes related to resource allocation and system design, thereby contributing to the ongoing discourse on efficient data management in contemporary computing environments.
\begin{itemize}
  \item Problem statement
  \begin{itemize}
    \item High data growth problem and dram capacity scaling decreases
    \item Big data frameworks require more memory
  \end{itemize}

  \item Possible approaches for more memory
  \begin{itemize}
    \item local storage device (e.g., NVMe SSD) or remote storage device or
      remote memory 
    \item Remote storage device or remote memory in case of disaggregated
      datacenter
    \item Storage device provide higher capacity than remote memory
    \item Remote memory has lower latency
    \item SPDK-nvme-of can reduce device latency
  \end{itemize}

  \item Evaluation of existing approaches in the context of managed big data
    frameworks 
  \item We use TeraHeap which is the state of the art

  \item {Evaluation + some high level results}
\end{itemize}
